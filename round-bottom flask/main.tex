\documentclass[tikz,border=10pt]{standalone}
\usepackage{graphicx}

\def\neckWidth{6}
\def\neckHeight{15}
\def\bodyWidth{19}
\def\bodyHeight{40}

\begin{document}
%\begin{tikzpicture}[x=1pt,y=1pt]
\begin{tikzpicture}[scale=0.5,line width=5pt]

% 丸底フラスコの首部分の左右の枠
\draw (\neckWidth,0) -- (\neckWidth+1,\neckHeight);
\draw (-\neckWidth,0) -- (-\neckWidth-1,\neckHeight);

% 丸底フラスコの首部分の上下の楕円
\draw (0,\neckHeight) ellipse ({\neckWidth+1} and 1);
\draw (0,\neckHeight-\neckHeight) ellipse ({\neckWidth} and 1);

% 丸底フラスコの胴部分の下半分の円弧
\draw[fill=blue!18] (-\bodyWidth+0.,-\bodyHeight/2-4) arc (180:360:{\bodyWidth-0.} and {\bodyHeight/2-1});

% 胴中心部の上凸円弧+グラデーション塗りつぶし
\shade [inner color=blue!6, outer color=blue!18] (-\bodyWidth+0.05,-\bodyHeight/2-4.05) arc (180:0:{\bodyWidth-0.05} and 3);

% 胴中心部の下凸円弧
\draw (-\bodyWidth+0.01,-\bodyHeight/2-4) arc (180:360:{\bodyWidth-0.01} and 3);

% 丸底フラスコの胴部分の上半分の枠
\draw (-\neckWidth,0)
.. controls (-\neckWidth,-6) and +(1,\bodyHeight/3+2)
.. (-\bodyWidth,-\bodyHeight/2-4);
\draw (\neckWidth,0)
.. controls (\neckWidth,-6) and +(-1,\bodyHeight/3+2)
.. (\bodyWidth,-\bodyHeight/2-4);

% フラスコ内部の白い円
\fill[white] (2,-\bodyHeight/2-12.5) circle [radius=2];
\fill[white] (-2,-\bodyHeight/2-15.5) circle [radius=1.2];
\fill[white] (1,-\bodyHeight/2-18.5) circle [radius=1.0];

\end{tikzpicture}
\end{document}